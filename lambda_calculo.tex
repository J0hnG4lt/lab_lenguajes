\documentclass{article}
\usepackage[utf8]{inputenc}
\usepackage{amsmath}

\begin{document}


\section*{}

\begin{center}
Universidad Simón Bolívar
\\Departamento de Computación y Tecnología de la Información
\\Asignatura: CI3661
\\Sección: 2
\\Nombre: Georvic Tur 
\\Carnet: 12-11402

\section*{Tarea 1}

\end{center}


\subsection*{Parte 1}

\begin{enumerate}

\item Consideremos que los tokens CONS, HEAD y TAIL son abreviaciones de los términos:

\begin{center}

$\lambda a . \lambda b . \lambda f . f\  a\  b$
\\$\lambda c . c (\lambda a . \lambda b . a)$
\\$\lambda c . c (\lambda a . \lambda b . b)$

\end{center}

respectivamente. Resduzca hasta su forma normal el término TAIL (CONS p q) mostrando paso a paso cada reducción.

\subsubsection*{Respuesta:}

\begin{align*}
&TAIL\ ( CONS\ p\ q )
\\&= TAIL\ ((\lambda a . \lambda b . \lambda f . f \  a \  b) \  p \  q)
\\&\rightarrow TAIL\  ((\lambda b . \lambda f . f \  p \  b)\  q)
\\&\rightarrow TAIL\  (\lambda f . f\ p\ q)
\\&= (\lambda c.c\ (\lambda a . \lambda b . b))\ (\lambda f . f\ p\ q)
\\&\rightarrow (\lambda f.f\ p\ q)\ (\lambda a.\lambda b.b)
\\&\rightarrow (\lambda a. \lambda b.b)\ (p\ q)
\\&\rightarrow (\lambda b.b)\ q
\\&\rightarrow \ q
\end{align*}


\item Si abreviamos por REST A al término:

\begin{center}
$
\lambda x_0 .\lambda x_1 . x_0 (\lambda x_0 .\lambda x_1 .\lambda x_2 . x0 (\lambda x_3 .\lambda x_4 . x_4 ( x_3 x_1 ))(\lambda x_5 . x_2 )(\lambda x_5 . x_5 )) x_1
$
\end{center}

y por 1 y 4 a los términos $ \lambda x_0 .\lambda x_1 . x_0 x_1 $ y $ \lambda x_0 . \lambda x_1 . x_0 ( x_0 ( x_0 ( x_0 x_1 ))) $ respectivamente, entonces calcule la forma normal del término REST A 1 4, mostrando paso a paso cada reducción.


\subsubsection*{Respuesta:}
\begin{align*}
&RESTA\ 1\ 4
\\&= \lambda x_0 .\lambda x_1 . x_0 (\lambda x_0 .\lambda x_1 .\lambda x_2 . x0 (\lambda x_3 .\lambda x_4 . x_4 ( x_3 x_1 ))(\lambda x_5 . x_2 )(\lambda x_5 . x_5 )) x_1\ 1\ 4
\\&\rightarrow \lambda x_1 . 1\ (\lambda x_0 .\lambda x_1 .\lambda x_2 . x0 (\lambda x_3 .\lambda x_4 . x_4 ( x_3 x_1 ))(\lambda x_5 . x_2 )(\lambda x_5 . x_5 )) x_1\ 4
\\&= \lambda x_1 . ( \lambda x_0 .\lambda x_1 . x_0 x_1)(\lambda x_0 .\lambda x_1 .\lambda x_2 . x0 (\lambda x_3 .\lambda x_4 . x_4 ( x_3 x_1 ))(\lambda x_5 . x_2 )(\lambda x_5 . x_5 )) x_1\ 4
\\&\rightarrow ( \lambda x_0 .\lambda x_1 . x_0 x_1)(\lambda x_0 .\lambda x_1 .\lambda x_2 . x0 (\lambda x_3 .\lambda x_4 . x_4 ( x_3 x_1 ))(\lambda x_5 . x_2 )(\lambda x_5 . x_5 ))\ 4
\\&\rightarrow (\lambda x_1 . ((\lambda x_0 .\lambda x_1 .\lambda x_2 . x0 (\lambda x_3 .\lambda x_4 . x_4 ( x_3 x_1 ))(\lambda x_5 . x_2 )(\lambda x_5 . x_5 ))) x_1)\ 4
\\&\rightarrow (((\lambda x_0 .\lambda x_1 .\lambda x_2 . x0 (\lambda x_3 .\lambda x_4 . x_4 ( x_3 x_1 ))(\lambda x_5 . x_2 )(\lambda x_5 . x_5 )))\ 4)
\\&\rightarrow (((\lambda x_1 .\lambda x_2 . 4\ (\lambda x_3 .\lambda x_4 . x_4 ( x_3 x_1 ))(\lambda x_5 . x_2 )(\lambda x_5 . x_5 ))))
\\&\rightarrow (((\lambda x_1 .\lambda x_2 . (\lambda x_0 . \lambda x_1 . x_0 ( x_0 ( x_0 ( x_0 x_1 ))))(\lambda x_3 .\lambda x_4 . x_4 ( x_3 x_1 ))(\lambda x_5 . x_2 )(\lambda x_5 . x_5 ))))
\\&\rightarrow (\lambda x_1 .\lambda x_2 .(\lambda x_1 . (\lambda x_3 .\lambda x_4 . x_4 ( x_3 x_1 ))\\&\ \ ( (\lambda x_3 .\lambda x_4 . x_4 ( x_3 x_1 )) ( (\lambda x_3 .\lambda x_4 . x_4 ( x_3 x_1 ))( (\lambda x_3 .\lambda x_4 . x_4 ( x_3 x_1 )) x_1 ))))\\&\ \ (\lambda x_5 . x_2 )(\lambda x_5 . x_5 ))
\\&\rightarrow (\lambda x_1 .\lambda x_2 . ((\lambda x_3 .\lambda x_4 . x_4 ( x_3 x_1 )) ( (\lambda x_3 .\lambda x_4 . x_4 ( x_3 x_1 )) ( (\lambda x_3 .\lambda x_4 . x_4 ( x_3 x_1 )) ( (\lambda x_3 .\lambda x_4 . x_4 ( x_3 x_1 ))\\&\ \ (\lambda x_5 . x_2 ) ))))(\lambda x_5 . x_5 ))
\\&\rightarrow (\lambda x_1 .\lambda x_2 .(\lambda x_4 . x_4 ( ( (\lambda x_3 .\lambda x_4 . x_4 ( x_3 x_1 )) ( (\lambda x_3 .\lambda x_4 . x_4 ( x_3 x_1 )) ( (\lambda x_3 .\lambda x_4 . x_4 ( x_3 x_1 ))(\lambda x_5 . x_2 ) ))) x_1 ))\\&\ \ (\lambda x_5 . x_5 ))
\\&\rightarrow (\lambda x_1 .\lambda x_2 .(\lambda x_5 . x_5 )(( (\lambda x_3 .\lambda x_4 . x_4 ( x_3 x_1 )) ( (\lambda x_3 .\lambda x_4 . x_4 ( x_3 x_1 )) ( (\lambda x_3 .\lambda x_4 . x_4 ( x_3 x_1 ))(\lambda x_5 . x_2 ) ))) x_1 ))
\\&\rightarrow (\lambda x_1 .\lambda x_2 .(((\lambda x_3 .\lambda x_4 . x_4 ( x_3 x_1 )) ( (\lambda x_3 .\lambda x_4 . x_4 ( x_3 x_1 )) ( (\lambda x_3 .\lambda x_4 . x_4 ( x_3 x_1 ))(\lambda x_5 . x_2 ) ))) x_1 ))
\\&\rightarrow (\lambda x_1 .\lambda x_2 .((\lambda x_4 . x_4 ( ( (\lambda x_3 .\lambda x_4 . x_4 ( x_3 x_1 )) ( (\lambda x_3 .\lambda x_4 . x_4 ( x_3 x_1 ))(\lambda x_5 . x_2 ) )) x_1 )) x_1 ))
\\&\rightarrow (\lambda x_1 .\lambda x_2 . (x_1 ( ( (\lambda x_3 .\lambda x_4 . x_4 ( x_3 x_1 )) ( (\lambda x_3 .\lambda x_4 . x_4 ( x_3 x_1 ))(\lambda x_5 . x_2 ) )) x_1 )))
\\&\rightarrow (\lambda x_1 .\lambda x_2 .(x_1 ((\lambda x_4 . x_4 ( ( (\lambda x_3 .\lambda x_4 . x_4 ( x_3 x_1 ))(\lambda x_5 . x_2 )) x_1 )) x_1 )))
\\&\rightarrow (\lambda x_1 .\lambda x_2 .(x_1(x_1(( (\lambda x_3 .\lambda x_4 . x_4 ( x_3 x_1 ))(\lambda x_5 . x_2 )) x_1 ) )))
\\&\rightarrow (\lambda x_1 .\lambda x_2 . (x_1(x_1(( \lambda x_4 . x_4 ((\lambda x_5 . x_2 ) x_1 ) ) x_1 ))))
\\&\rightarrow (\lambda x_1 .\lambda x_2 .x_1(x_1(x_1 ((\lambda x_5 . x_2 ) x_1 ))) )
\\&\rightarrow (\lambda x_1 .\lambda x_2 . x_1( x_1( (x_1\ x_2 ))))
\\&= (\lambda x_0 .\lambda x_1 . x_0( x_0( (x_0\ x_1 ))))
\\&=\ 3
\end{align*}

\end{enumerate}


\subsection*{Parte 2}

\begin{enumerate}

\item Basado en el marco teórico expuesto anteriormente, plantee un esquema de recursión para la función de Fibonacci (que abeviaremos FIB) usando las operaciones IF , +, \textasteriskcentered, =.

\subsubsection*{Respuesta:}
Se puede usar el siguiente esquema:
\begin{align*}
&FIB=(\lambda n . FIB'\ 0\ 1\ 0\ n)
\\&FIB'=(\lambda a . \lambda b . \lambda j . \lambda n . IF\ (=\ j\ n)\ b\ (FIB'\ b\ (+\ a\ b)\ (+\ j\ 1)\ n))
\end{align*}




\item Usando el $\lambda-t \acute ermino$ y determine un $\lambda-t \acute ermino$ válido que compute la función de Fibonacci

\begin{align*}
&FIB=(\lambda n . FIB'\ 0\ 1\ 0\ n)
\\&FIB'=(\lambda x . \lambda a . \lambda b . \lambda j . \lambda n . IF\ (=\ j\ n)\ b\ (x\ b\ (+\ a\ b)\ (+\ j\ 1)\ n))\ FIB'
\end{align*}
lo cual se puede escribir como:
\begin{align*}
&FIB'=H\ FIB'
\\&donde
\\& H=(\lambda x . \lambda a . \lambda b . \lambda j . \lambda n . IF\ (=\ j\ n)\ b\ (x\ b\ (+\ a\ b)\ (+\ j\ 1)\ n))
\end{align*}

Como $Y\ H' = H'\ (Y\ H')$ se satisface para todo $H'$ si $Y=(\lambda x_0 . (\lambda x_1 . x_0\ (x_1\ x_1))(\lambda x_1 . x_0\ (x_1\ x_1)))$, se tiene que:

\begin{align*}
&Y\ H\ = Y\ (\lambda x . \lambda a . \lambda b . \lambda j . \lambda n . IF\ (=\ j\ n)\ b\ (x\ b\ (+\ a\ b)\ (+\ j\ 1)\ n))
\end{align*}

es el término FIB buscado.

\subsubsection*{Respuesta:}


\item Usando el $\lambda-t \acute ermino$ que obtuvo en la pregunta anterior, calcule FIB 2 haciendo todas las reducciones correspondientes paso a paso.

\subsubsection*{Respuesta:}

\begin{align*}
&FIB\ 2
\\&=(\lambda n . FIB'\ 0\ 1\ 0\ n)\ 2
\\&\rightarrow FIB'\ 0\ 1\ 0\ 2
\\&=(\lambda x . \lambda a . \lambda b . \lambda j . \lambda n . IF\ (=\ j\ n)\ b\ (x\ b\ (+\ a\ b)\ (+\ j\ 1)\ n))\ FIB'\ 0\ 1\ 0\ 2
\\&\rightarrow (\lambda a . \lambda b . \lambda j . \lambda n . IF\ (=\ j\ n)\ b\ (FIB'\ b\ (+\ a\ b)\ (+\ j\ 1)\ n))\ 0\ 1\ 0\ 2
\\&\rightarrow (\lambda b . \lambda j . \lambda n . IF (=\ j\ n)\ b\ (FIB'\ b\ (+\ 0\ b)\ (+\ j\ 1)\ n))\ 1\ 0\ 2
\\&\rightarrow (\lambda j . \lambda n . IF\ (=\ j\ n)\ b\ (FIB'\ 1\ (+\ 0\ 1)\ (+\ j\ 1)\ n))\ 0\ 2
\\&\rightarrow (\lambda n . IF\ (=\ 0\ n)\ b\ (FIB'\ 1\ (+\ 0\ 1)\ (+\ 0\ 1)\ n))\ 2
\\&\rightarrow (IF\ (=\ 0\ 2)\ b\ (FIB'\ 1\ (+\ 0\ 1)\ (+\ 0\ 1)\ 2))
\\&\rightarrow (IF\ (FALSE)\ b\ (FIB'\ 1\ (+\ 0\ 1)\ (+\ 0\ 1)\ 2))
\\&\rightarrow (FIB'\ 1\ (+\ 0\ 1)\ (+\ 0\ 1)\ 2)
\\&\rightarrow (FIB'\ 1\ 1\ (+\ 0\ 1)\ 2)
\\&\rightarrow (FIB'\ 1\ 1\ 1\ 2)
\\&=(\lambda x . \lambda a . \lambda b . \lambda j . \lambda n . IF\ (=\ j\ n)\ b\ (x\ b\ (+\ a\ b)\ (+\ j\ 1)\ n))\ FIB'\ 1\ 1\ 1\ 2
\\&\rightarrow(\lambda a . \lambda b . \lambda j . \lambda n . IF\ (=\ j\ n)\ b\ (FIB'\ b\ (+\ a\ b)\ (+\ j\ 1)\ n))\ 1\ 1\ 1\ 2
\\&\rightarrow(\lambda b . \lambda j . \lambda n . IF\ (=\ j\ n)\ b\ (FIB'\ b\ (+\ 1\ b)\ (+\ j\ 1)\ n))\ 1\ 1\ 2
\\&\rightarrow(\lambda j . \lambda n . IF\ (=\ j\ n)\ 1\ (FIB'\ 1\ (+\ 1\ 1)\ (+\ j\ 1)\ n))\ 1\ 2
\\&\rightarrow(\lambda n . IF\ (=\ 1\ n)\ 1\ (FIB'\ 1\ (+\ 1\ 1)\ (+\ 1\ 1)\ n))\ 2
\\&\rightarrow(IF\ (=\ 1\ 2)\ 1\ (FIB'\ 1\ (+\ 1\ 1)\ (+\ 1\ 1)\ 2))
\\&\rightarrow(IF\ (FALSE)\ 1\ (FIB'\ 1\ (+\ 1\ 1)\ (+\ 1\ 1)\ 2))
\\&\rightarrow (FIB'\ 1\ (+\ 1\ 1)\ (+\ 1\ 1)\ 2)
\\&\rightarrow (FIB'\ 1\ 2\ (+\ 1\ 1)\ 2)
\\&\rightarrow (FIB'\ 1\ 2\ 2\ 2)
\\&=(\lambda x . \lambda a . \lambda b . \lambda j . \lambda n . IF (=\ j\ n)\ b\ (x\ b\ (+\ a\ b)\ (+\ j\ 1)\ n))\ FIB'\ 1\ 2\ 2\ 2
\\&\rightarrow(\lambda a . \lambda b . \lambda j . \lambda n . IF\ (=\ j\ n)\ b\ (FIB'\ b\ (+\ a\ b)\ (+\ j\ 1)\ n))\ 1\ 2\ 2\ 2
\\&\rightarrow(\lambda b . \lambda j . \lambda n . IF\ (=\ j\ n)\ b\ (FIB'\ b\ (+\ 1\ b)\ (+\ j\ 1)\ n))\ 2\ 2\ 2
\\&\rightarrow(\lambda j . \lambda n . IF\ (=\ j\ n)\ 2\ (FIB'\ 2\ (+\ 1\ 2)\ (+\ j\ 1)\ n))\ 2\ 2
\\&\rightarrow(\lambda n . IF\ (=\ 2\ n)\ 2\ (FIB'\ 2\ (+\ 1\ 2)\ (+\ 2\ 1)\ n))\ 2
\\&\rightarrow(IF\ (=\ 2\ 2)\ 2\ (FIB'\ 2\ (+\ 1\ 2)\ (+\ 2\ 1)\ 2))
\\&\rightarrow(IF\ (TRUE)\ 2\ (FIB'\ 2\ (+\ 1\ 2)\ (+\ 2\ 1)\ 2))
\\&\rightarrow\ 2
\end{align*}

\end{enumerate}

\end{document}
